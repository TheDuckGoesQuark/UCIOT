%%%%%%%%%%%%%%%%%%%%%%%%%%%%%%%%%%%%%%%%%
% University Assignment Title Page 
% LaTeX Template
% Version 1.0 (27/12/12)
%
% This template has been downloaded from:
% http://www.LaTeXTemplates.com
%
% Original author:
% WikiBooks (http://en.wikibooks.org/wiki/LaTeX/Title_Creation)
%
% License:
% CC BY-NC-SA 3.0 (http://creativecommons.org/licenses/by-nc-sa/3.0/)
%
%%%%%%%%%%%%%%%%%%%%%%%%%%%%%%%%%%%%%%%%%
%\title{Title page with logo}
%----------------------------------------------------------------------------------------
%	PACKAGES AND OTHER DOCUMENT CONFIGURATIONS
%----------------------------------------------------------------------------------------

\documentclass[12pt]{article}
\usepackage[english]{babel}
\usepackage[utf8x]{inputenc}
\usepackage{natbib}
\usepackage{amsmath}
\usepackage[colorinlistoftodos]{todonotes}
\usepackage{listings}
\usepackage{color}
\usepackage{tabu}
\tabulinesep=1.2mm
\usepackage[explicit]{titlesec}
\usepackage[hyphens,spaces,obeyspaces]{url}
\usepackage{subfig}
\usepackage{graphicx}
\usepackage{grffile}
\usepackage{mwe}
\usepackage[section]{placeins}

\begin{document}

\begin{titlepage}

\newcommand{\HRule}{\rule{\linewidth}{0.5mm}} % Defines a new command for the horizontal lines, change thickness here

\center % Center everything on the page
 
%----------------------------------------------------------------------------------------
%	HEADING SECTIONS
%----------------------------------------------------------------------------------------

\textsc{\LARGE University of St Andrews}\\[1.5cm] % Name of your university/college
\textsc{\Large CS4099}\\[0.5cm] % Major heading such as course name
\textsc{\large }\\[0.5cm] % Minor heading such as course title

%----------------------------------------------------------------------------------------
%	TITLE SECTION
%----------------------------------------------------------------------------------------

\HRule \\[0.4cm]
{ \huge \bfseries ILNP Routing for IoT}\\[0.4cm] % Title of your document
\HRule \\[1.5cm]
 
%----------------------------------------------------------------------------------------
%	AUTHOR SECTION
%----------------------------------------------------------------------------------------


\Large \emph{Author:}\\
 \textsc{Jordan Mackie}\\[1cm] % Your name
 
\Large \emph{Supervisor:}\\
 \textsc{Prof Saleem Bhatti}\\[1cm] % Your name
%----------------------------------------------------------------------------------------
%	DATE SECTION
%----------------------------------------------------------------------------------------

{\large \today}\\[2cm] % Date, change the \today to a set date if you want to be precise

%----------------------------------------------------------------------------------------
%	LOGO SECTION
%---------------------------------------------------------------------------------------

\includegraphics[width = 2.5cm]{images/standrewslogo.png}
 
%----------------------------------------------------------------------------------------

\vfill % Fill the rest of the page with whitespace

\end{titlepage}

\pagenumbering{gobble}

\section*{Abstract}



\section*{Declaration}
I declare that the material submitted for
assessment is my own work except where credit is
explicitly given to others by citation or
acknowledgement. This work was performed during
the current academic year except where otherwise
stated.
The main text of this project report is \#TODO NN,NNN
words long, including project specification and plan.
In submitting this project report to the University of
St Andrews, I give permission for it to be made
available for use in accordance with the regulations of
the University Library. I also give permission for
the title and abstract to be published and for copies of
the report to be made and supplied at cost to any bona
fide library or research worker, and to be made
available on the World Wide Web. I retain the
copyright in this work.

\newpage

\tableofcontents

\newpage
\pagenumbering{arabic}
\setcounter{page}{1} 

\section{Introduction}

Despite the imminent exhaustion of IPv4 addresses \cite{ripe_labs}, IPv6 is still being adopted slowly \cite{google_ipv6}. Brittle solutions such as NAT are being used to temporarily expand the IP address space , and to avoid the transition costs involved in upgrading to IPv6. Whilst IPv6 does expand the address space greatly and introduces functionality such as multicast, the internet protocol itself suffers from many issues. 

\subsection{Issues with IP}

IP addresses are used both to identify a system and to determine its topological location. \cite{briancarpenter2014} lists several of the downsides to this overloading of IP addresses, and why the protocol was still used despite these concerns. 

The separation of concerns that should be achieved by a layered model is not possible, since the IP address is used by the each layer in some way. IP addresses can be used in the application layer, and are bound to physical network interfaces, which goes against the end-to-end argument where each layer should provide a opaque abstraction to those above it.

The issues with IP are not just semantic. Due to the overloading of the IP address and the rapid increase in internet connected devices \cite{iot_stat}, the scalability of the system is being challenged. Implementations of multipath routing with the intention of balancing load is improving network performance for the operators that use them, but with IP it places greater stress on the default-free zone (DFZ) routing information base (RIB). Multihoming is also being used increasingly to improve reliability, but with IP this requires routing entries to store multiple addresses for one host. An IAB workshop \cite{rfc4984} detailed how the DFZ RIB databases are growing in size exponentially due to the increasing number of devices and an inability to aggregate address prefixes. With IPv6 allowing for an even larger address space, this problem will only get worse. 

Due to the growing number of Internet of Things (IoT) devices, mobility is also a necessary feature for a networking protocol. Mobile IP currently requires another entity (a home agent) to track and proxy packets to the mobile host as it changes networks. This mobility is also problematic for IPSec, which requires that the end system addresses remain fixed.

Given the difficulty involved in simply migrating from IPv4 to IPv6, it is very doubtful that introducing an entirely different protocol for the internet would be successful. A backwards compatible solution would likely be the only solution that would be adopted within a reasonable time frame.

\subsection{ILNP}

Both multihoming and mobility are far simpler to implement and maintain if the identity and topological locator of a host are separated, and this is how the Identifier-Locator Network Protocol functions. \cite{5586444} proposes ILNPv6, which implement ILNP with the same address space as IPv6.

\subsection{Goal}


\section{Context Survey}

\begin{enumerate}
\item ILNP research 
\item Ad Hoc sensor networks
\item Energy effecient routing protocols
\end{enumerate}

\section{Requirements Specification}

\begin{enumerate}
\item Describe requirements of resulting python library
\end{enumerate}

\section{Design}

\begin{enumerate}
\item Component structure (socket interface, router/dsrservice/forwardingtable, raw sockets)
\item Runtime behaviour (packet parsing, routing, and forwarding)
\item Use figures to visualise project structure and workflow
\end{enumerate}

\section{Experiment}

\begin{enumerate}
\item Discuss aim of experiment (to measure effeciency of the used routing protocol with ILNP, and compare to IP).
\item Explain case study, with reference to source (i.e. agricultural sensor setup)
\item Use visuals to show locators to real life position and sensor radi
\item Discuss experiment configuration (how machines were chosen, results collected, battery life simulated, etc)
\item discuss choice of metrics, justifcation and how to compare results.
\end{enumerate}

\section{Results and Discussion}

\begin{enumerate}
\item Show heat map of results
\item Explain features of heat map
\item Describe the behaviour if IP was used instead through analysis
\item Discuss weaknesses with experiment
\end{enumerate}

\section{Conclusions}

\begin{enumerate}	
\item was the goal met, and if so how well?
\item future work with ILNP, possible suggestions of better alternatives to the routing protocol used.
\end{enumerate}

\section{Appendix}
\begin{enumerate}
\item Instructions on installing, and executing and using the python module, and how to configure the experiments.
\end{enumerate}

\bibliographystyle{unsrt}
\bibliography{mybib}

\end{document}
